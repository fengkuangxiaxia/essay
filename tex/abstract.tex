%%==================================================
%% abstract.tex for SJTU Master Thesis
%%==================================================

\begin{abstract}

随着社会的信息化,地球已成为地球村,人与人之间的距离缩短、联系更加紧密。而现在,物联网作为一种新的网络模式,也渐渐进入人们的实现,并越来越得到重视。从宏观上讲,我国已经把发展研究物联网作为战略上的要求;从微观上讲,现在市场上开始出现的一些智能家居,以及现在的车联网都是物联网的一种体现。虽然物联网现在的发展和研究都仍处于起步阶段,对于物联网的各种标准都尚未有明确规定,导致各厂商之间都是按照自己的标准行事,使得物联网这个产业其实一定程度上很混乱。然而不论日后对于物联网会有什么标准,必然都会包含安全这一方面的标准。由于互联网的各种安全事件的发生,我们已经见识到安全是多么的重要,因此在物联网这种与生活更紧密结合的模式中必定对于安全有更严格的要求。而Trivium则可能是适应未来物联网时代的一种算法。
Trivium算法作为一种基于硬件的流密码算法,是欧洲流密码工程eSTREAM的最终胜选算法之一,由Christophe De Canniere和Bart Preneel提出,在其出现之后就有在密码界引起了关注,因为其设计简单优美,而且也宣传它具有很强的安全性,因此只要其安全性得到验证,就十分适合用来作为物联网中的密码算法。

\keywords{\large Trivium \quad 物联网 \quad 流密码}
\end{abstract}

\begin{englishabstract}



\englishkeywords{\large Trivium, Internet of Things, stream ciphers}
\end{englishabstract}

