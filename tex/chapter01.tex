%%==================================================
%% chapter01.tex for SJTU Master Thesis
%%==================================================

\chapter{绪论}
\label{chap:introduction}

\section{Trivium算法的简单介绍}

Trivium算法是一种基于硬件的流密码算法,是欧洲流密码工程eSTREAM的最终胜选算法之一。其设计初衷是想试验是否存在一种简单的流密码算法能够兼具高效性、可变性以及最重要的安全性。

由于物联网中各种芯片的逻辑门数肯定会有所限制,因此可能无法实现复杂的加密算法,所以作为流密码的Trivium似乎是一种不错的选择。因此Trivium算法的安全性和效率直接关系到未来的物联网中是否能使用Trivium算法作为一种加密标准被广泛采用。因此本文将对Trivium的安全性和效率进行研究。

通过对Trivium算法的安全性和效率的研究,本文著者希望给出在物联网中Trivium算法是否是合适的加密算法的判断,并且对Trivium算法加以改进,以使其更适合物联网的使用环境。

具体地,本文研究了Trivium算法以及具有与Trivium算法相似结构的Trivium型算法,计算了Trivium型算法的内部状态位数的下限,并构造了几种能产生小周期的初始内部状态,通过分析构造方法,分析Trivium算法及Trivium型算法中的具体使用步骤是否安全。


\section{物联网发展与现状}

物联网这一概念出现至今已超过10年时间,然而仍然没有一个统一且明确的定义。

1999年,MITAutoIDCenter给出了早期的“物联网”定义:在计算机互联网的基础上,利用RFID、无线数据通信等技术,构造一个覆盖世界上各种事物的网络(InternetofThings),以实现自动识别物品和互联共享信息\parencite{宁焕生2010全球物联网发展及中国物联网建设若干思考}。

2005年,国际电信联盟(ITU)发布的《ITU互联网报告2005:物联网》中正式给出了“物联网”概念并对扩展了其涵义,指出物联网是互联网应用的延伸,实现物联网的四大核心技术将是“RFID、传感器技术、纳米技术、智能嵌入技术”\parencite{宁焕生2010全球物联网发展及中国物联网建设若干思考}。

2009年9月,欧盟公布的一份CERPIoTSRA中,将“物联网”定义为:物联网将是未来互联网不可分割的一部分,是一个动态的全球网络架构,它具备基于一定的标准和互用的通信协议的自组织能力.其中物理的和虚拟的“物”均具有身份标识、物理属性和虚拟特性,并应用智能接口可以无缝链接到信息网络\parencite{宁焕生2010全球物联网发展及中国物联网建设若干思考}。

自2009年IBM提出“智慧地球”这一概念后,物联网在全球掀起了一股浪潮,美国、欧洲、我国都把物联网作为战略发展计划。然而物联网的发展却仍只是停留在起步阶段,且不论上述的对于物联网这一概念都尚未有明确统一的定义,对于物联网中要用的标示码、各种标准各国之间哪怕是一个国家内的不同厂商也尚未达成一致,因此要有一个严格明确的物联网实际应用实验环境还早的很。现在市场上的物联网相关产品多数是车联网或者智能家居或者物流方面的相关应用,然而这距离物物相连、自动化处理仍很遥远,可以说物联网的现状仍然是处于制定标准阶段以及一些小范围的实验性产品。

\section{本文的研究意义和研究成果}

Trivium算法作为一种新的算法,虽然得到了密码界的大量关注,但发表的对于Trivium算法的相关研究却并不多,其中更是几乎都是对于Trivium算法的攻击研究,而鲜有对Trivium算法的实际应用做研究探讨的。

另外现在的物联网的研究很多都集中在研究分析物联网的发展方向或者体系结构,又或者是将各种现有系统套上物联网,然后研究是否可行或者有什么优劣,却鲜有研究物联网下的安全问题的,因此研究物联网下的密码算法是一项有意义的工作。另外现在的各种加密手段很多都是用RSA、AES、3DES这种,在互联网中,考虑到设备的计算能力,这些都可以接受,但是在物联网时代,是否所有物联网设备都支持就不得而知了,因此研究物联网下的轻量级密码算法Trivium算法是有意义的。

本文的创新工作在于计算了与Trivium算法相似结构的Trivium型算法的内部状态位数的下限,另外给出了构造能产生小周期的初始内部状态的方法。以及提出了将Trivium算法应用到物联网场景下的设计想法。
