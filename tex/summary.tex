%%==================================================
%% conclusion.tex for SJTUThesis
%% Encoding: UTF-8
%%==================================================

\begin{summary}
本文以欧洲eSTREAM计划的最终获胜算法之一的Trivium算法为研究对象。在第一章中对Trivium算法和物联网进行简单的介绍。

在第二章中本文首先对Trivium算法的结构进行研究,通过对Trivium算法的状态转移矩阵和由状态转移矩阵计算出的特征多项式的研究,将标准Trivium算法参数化,定义了Trivium型算法,并找到了满足Trivium型算法定义下,内部状态位的最小值及此时所有的参数取法。接着本文通过Bivium和Trivium的关系,将Trivium型算法拓展为n阶Trivium型算法。接着本文为了便于研究,取了一组内部状态位仅为42位(同时也是满足Trivium型算法定义下的最小内部状态位数)的Trivium型算法为研究对象,并通过手工计算的方式,算出了该Trivium型算法经过三轮后内部状态如何由初始的内部状态构成,为了得到周期为3的解,以此得到了42个方程组成的方程组,经过化简最终得到了该Trivium型算法能使周期为3的初始内部状态的构造方法,并通过对3这一特殊周期的分析,探讨了经过3轮后任意Trivium型算法如果要有周期3则对应的方程组的规律,之后将周期为3的初始内部状态构造方法推广至所有Trivium型算法,然后再进一步推广至n阶Trivium算法,给出了n阶Trivium算法下周期为3的初始内部状态的个数和具体构造方法。

在第三章中,本文对是否能将Trivium算法运用到物联网中进行了研究。本文首先分析了Trivium算法在物联网环境下相较于其他加密算法的优势。具体为对Trivium算法的高效性和安全性进行研究。然后本文以贵重物品追踪为假想的应用环境,将Trivium算法应用于其中进行设计。首先对于贵重物品追踪这一概念的背景进行了简单的介绍,然后分析了定位方案,对GPS和RFID的选择进行了分析。然后对现有的基本RFID协议进行简单介绍,包括Hash-Lock协议、随机化的Hash-Lock协、Hash链协议、基于Hash的ID变化协议、David的数字图书馆RFID协议、分布式RFID询问-应答认证协议、LCAP协议,并从中选择了一种适合用于改造成Trivium算法的协议,且给出了Trivium版RFID的示意图和具体步骤。在设计完协议后,对该协议是否能抵抗重放攻击和欺骗攻击进行了分析,并通过Ban逻辑对设计的新协议验证其安全性并得出新协议能实现标签与阅读器双向认证的结论。最后本文给出贵重物品追踪这一假想应用场景下,整个架构图和安全性分析。

然而本文仍有许多地方没有顾及。首先,在对标准Trivium算法的研究中,本文虽然对线性部分的状态转移矩阵和对应的特征多项式进行了计算和研究,但是并没有具体给出线性部分的周期分析。其次,本文在研究能产生小周期的初始内部状态时,虽然具体给出了能产生周期为3的初始内部状态的构造方法并推广到Trivium算法、Trivium型算法、n阶Trivium算法,并给出为何周期为2和7为何无解的分析,但是实际上在跑数据的过程中,本文著者发现除了能经常找到产生周期为3的初始内部状态外还能找到能产生周期为$3(2^{m}-1)$的初始内部状态,这里m为本原多项式g(x)最高次的次数,虽然著者得到了这个结果,并猜测这个值是线性部分的最大周期,但虽然能给出粗略的寻找方法但却无法给出具体的构造方法,并且如果通过粗略的寻找方法要找到所有能产生该周期的初始内部状态的计算量能很大,因此本文中并没给出描述。最后,本文虽然给出了RFID的Trivium版本并给出了贵重物品追踪的设计图,但并没有实际实现,因此对于本设计中可能产生的实际问题,本文并没有研究也无法进行研究。

针对以上这些问题,著者对本研究有如下展望。首先对Trivium的线性部分周期希望能研究出具体的周期,最好能够给出不同初始内部状态对应的线性部分周期。其次,研究$3(2^{m}-1)$是否为线性部分的周期上限,并且如何能像构造产生周期为3的初始内部状态一样给出具体的构造产生周期为$3(2^{m}-1)$的初始内部状态的方法,而不是给出计算量仍很大的粗略寻找方法,最好能像周期为3一样推广至n阶Trivium型算法。最后,著者希望能在实际工程项目中,将Trivium算法应用进去,查看Trivium算法在真实工程项目的表现。

\end{summary}
