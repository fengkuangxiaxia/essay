%%==================================================
%% conclusion.tex for SJTUThesis
%% Encoding: UTF-8
%%==================================================

\begin{summary}
本文以欧洲eSTREAM计划的最终获胜算法之一的Trivium算法为研究对象。本文首先对Trivium算法的结构进行研究,通过将Trivium算法参数化,定义了Trivium型算法,并找到了满足Trivium型算法定义下,内部状态位的最小值及此时所有的参数取法。接着本文为了便于研究,取了一组内部状态位仅为42位的Trivium型算法为研究对象,并通过手工计算的方式,算出了该Trivium型算法经过三轮后内部状态如何由初始的内部状态构成,为了得到周期为3的解,以此得到了42个方程组成的方程组,经过化简最终得到了该Trivium型算法能使周期为3的初始内部状态的构造方法,并将其推广至所有Trivium型算法。

在之后,本文对是否能将Trivium算法运用到物联网中进行了研究。本文首先分析了Trivium算法在物联网环境下相较于其他加密算法的优势。然后以贵重物品追踪为假想的应用环境,将Trivium算法应用于其中进行设计。首先分析了定位方案,然后对现有的基本RFID协议进行描述,并从中选择了一种适合用于改造成Trivium算法的协议。在设计完协议后,通过Ban逻辑对设计的新协议验证其安全性并得出新协议能实现标签与阅读器双向认证的结论。最后本文给出贵重物品追踪这一假想应用场景下,整个设计图和安全性分析。

本文认为将Trivium应用到物联网中仍有很多值得研究,由于本文仅通过假想应用场景的方式分析Trivium算法在物联网中应用的优劣,因此如果有可能,本文可以进一步研究将该设计实际实现,并在实际应用中更详尽地分析Trivium算法在真实应用环境下的表现和改进方向。

\end{summary}
